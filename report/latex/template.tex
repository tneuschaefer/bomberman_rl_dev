%%%%%%%%%%%%%%%%%%%%%%%%%%%%%%%%%%%%%%%%%
% University/School Laboratory Report
% LaTeX Template
% Version 4.0 (March 21, 2022)
%
% This template originates from:
% https://www.LaTeXTemplates.com
%
% Authors:
% Vel (vel@latextemplates.com)
% Linux and Unix Users Group at Virginia Tech Wiki
%
% License:
% CC BY-NC-SA 4.0 (https://creativecommons.org/licenses/by-nc-sa/4.0/)
%
%%%%%%%%%%%%%%%%%%%%%%%%%%%%%%%%%%%%%%%%%

%----------------------------------------------------------------------------------------
%	PACKAGES AND DOCUMENT CONFIGURATIONS
%----------------------------------------------------------------------------------------

\documentclass[
	letterpaper, % Paper size, specify a4paper (A4) or letterpaper (US letter)
	10pt, % Default font size, specify 10pt, 11pt or 12pt
]{CSUniSchoolLabReport}

\addbibresource{references.bib} % Bibliography file (located in the same folder as the template)

%----------------------------------------------------------------------------------------
%	REPORT INFORMATION
%----------------------------------------------------------------------------------------

\title{\textbf{Reinforcement Learning for Bomberman \\ Final Project}} % Report title

\author{Angelina Basova \\ Tobias Neuschäfer \\Sven Zelch} % Author name(s), add additional authors like: '\& James \textsc{Smith}'

\date{\today} % Date of the report

%----------------------------------------------------------------------------------------

\begin{document}
\maketitle % Insert the title, author and date using the information specified above

\begin{center}
	\begin{tabular}{l r}
		Tutor:      & TODO                     \\
		Instructor: & \textsc{Ullrich K\"othe} % Instructor/supervisor
	\end{tabular}
\end{center}



\newpage

\tableofcontents
\newpage

\listoffigures
\listoftables
\newpage

% If you need to include an abstract, uncomment the lines below
%\begin{abstract}
%	Abstract text
%\end{abstract}

%----------------------------------------------------------------------------------------
%	INTRODUCTION
%----------------------------------------------------------------------------------------

\section{Introduction}

%----------------------------------------------------------------------------------------
%	METHODS
%----------------------------------------------------------------------------------------

\section{Methods}

\subsection{Q-Learning}
\subsection{Reward Shaping}

%----------------------------------------------------------------------------------------
%	TRAINING
%----------------------------------------------------------------------------------------

\section{Training}
\subsection{Q-Learning}

\subsubsection*{Q Table}
A major challenge to implement Q-Learning was the creation of a feature space.
We initialize the Q table as a dictionary, whose keys are the features and the values are the values of the Q function for the six
actions.
Choosing a dictionary as the preferred data structure for the Q table provided us
two major benefits.
First, we we were able to increase its size and thus its feature space flexibly.
This proved to be quite helpful to derive training strategies because we
could focus on the q values of almost the same states.
Additionally, we didn't feel the need to create a Q table covering every possible
feature space as for unseen states, we picked actions from a similar seen state.

\subsubsection*{Feature space}
Our features consist of ten values. The majority of the values contain the Manhattan distance between
the agent and the nearest object of interest. For instance, Figure \ref{fig:example} illustrates a possible feature.
The feature shows that the agent is only one cell away from a coin in the x-axis as the first value equals to one.
The y value equals to 0, which means that the agent is in the same row as the coin.


\begin{center}
	\begin{figure}[H]
		\begin{tabular}{ccccccccccc}
			          &                                                        &         &       &        &       &                                                         & \multicolumn{2}{c}{\includegraphics[width=1cm]{Figures/robot_pink.png}} &                              \\
			          & \multicolumn{2}{c}{\includegraphics{Figures/coin.png}} &         &       &        &       & \multicolumn{2}{c}{\includegraphics{Figures/crate.png}} & \multicolumn{2}{c}{\includegraphics{Figures/bomb_blue.png}}                                            \\
			\\
			feature = & (coin x,                                               & coin y, & left, & right, & down, & up,                                                     & crate x ,                                                               & crate y, & bomb x, & bomb y)
		\end{tabular}
		\caption{Illustration of the feature structure encoding the space of the agent}
	\end{figure}
\end{center}


\begin{center}
	\begin{figure}[h]
		\centering
		\begin{tabular}{ccccccccccc}
			feature = & (1, & 0, & 0, & 0, & 1, & 1, & 5, & 1, & 9, & 12)
		\end{tabular}
		\caption{Example of a feature }
		\label{fig:example}
	\end{figure}
\end{center}



\subsubsection*{Auxilliary Rewards}
In order to incentivize the agent to perform beneficial actions we used a number of auxilliary rewards.
The rewards were split into five different categories, which indicate in which case the auxilliary reward
is earned. For instance, the auxilliary reward MOVED\_IN\_CYCLE corresponds to the general category. This
means that the reward can be earned at any step in the game. While the auxilliary reward MOVED\_TO\_CRATE can
only be earned when a crate or an opponent is present in the features, thus its category is Crates, Opponents.

\begin{table}[h]
	\begin{center}
		\begin{tabular}{lll}
			Category & Auxilliary Rewards       & Value \\
			\hline \hline
			\multirow{4}*{General}
			         & e.VALID\_ACTION          &       \\
			         & NOT\_VALID\_ACTION       &       \\
			         & MOVED\_IN\_CYCLE         & on    \\
			         & NOT\_MOVED\_IN\_CYCLE    & on    \\
			\hline
			\multirow{4}*{Coins}
			         & e.COIN\_COLLECTED        &       \\
			         & e.COIN\_FOUND            &       \\
			         & MOVED\_TO\_COIN          & on    \\
			         & MOVED\_AWAY\_FROM\_COIN  & on    \\
			\hline
			\multirow{3}*{Crates, Opponents}
			         & e.DIDNT\_DROP\_BOMB      &       \\
			         & MOVED\_TO\_CRATE         & on    \\
			         & MOVED\_AWAY\_FROM\_CRATE & on    \\
			\hline
			\multirow{6}*{Bombs}
			         & MOVED\_TO\_BOMB          & on    \\
			         & MOVED\_AWAY\_FROM\_BOMB  & on    \\
			         & MEANINGFUL\_WAIT         &       \\
			         & NOT\_MEANINGFUL\_WAIT    &       \\
			         & SAFE\_FROM\_BOMB         &       \\
			         & NOT\_SAFE\_FROM\_BOMB    &       \\
			\hline
			\multirow{3}*{Life}
			         & e.KILLED\_SELF           &       \\
			         & NOT\_KILLED\_SELF        &       \\
			         & e.KILLED\_OPPONENT       &       \\
			         & e.OPPONENT\_ELIMINATES   &       \\
			         & e.GOT\_KILLED            &       \\
			         & e.SURVIVER\_ROUND        &       \\
			\hline
		\end{tabular}
		\caption{Auxilliary rewards and their values}
		\label{tab:rewards}
	\end{center}
\end{table}




%----------------------------------------------------------------------------------------
%	EXPERIMENTS
%----------------------------------------------------------------------------------------

\section{Experiments and Results }


%----------------------------------------------------------------------------------------
%	CONCLUSION
%----------------------------------------------------------------------------------------

\section{Conclusions}




%----------------------------------------------------------------------------------------
%	BIBLIOGRAPHY
%----------------------------------------------------------------------------------------

\printbibliography % Output the bibliography

%----------------------------------------------------------------------------------------

\end{document}