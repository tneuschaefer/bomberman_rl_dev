%%%%%%%%%%%%%%%%%%%%%%%%%%%%%%%%%%%%%%%%%
% University/School Laboratory Report
% LaTeX Template
% Version 4.0 (March 21, 2022)
%
% This template originates from:
% https://www.LaTeXTemplates.com
%
% Authors:
% Vel (vel@latextemplates.com)
% Linux and Unix Users Group at Virginia Tech Wiki
%
% License:
% CC BY-NC-SA 4.0 (https://creativecommons.org/licenses/by-nc-sa/4.0/)
%
%%%%%%%%%%%%%%%%%%%%%%%%%%%%%%%%%%%%%%%%%

%----------------------------------------------------------------------------------------
%	PACKAGES AND DOCUMENT CONFIGURATIONS
%----------------------------------------------------------------------------------------

\documentclass[
	letterpaper, % Paper size, specify a4paper (A4) or letterpaper (US letter)
	10pt, % Default font size, specify 10pt, 11pt or 12pt
]{CSUniSchoolLabReport}

\addbibresource{references.bib} % Bibliography file (located in the same folder as the template)

%----------------------------------------------------------------------------------------
%	REPORT INFORMATION
%----------------------------------------------------------------------------------------

\title{\textbf{Reinforcement Learning for Bomberman \\ Final Project}} % Report title

\author{Angelina Basova \\ Tobias Neuschäfer \\Sven Zelch} % Author name(s), add additional authors like: '\& James \textsc{Smith}'

\date{\today} % Date of the report

%----------------------------------------------------------------------------------------

\begin{document}
\maketitle % Insert the title, author and date using the information specified above

\begin{center}
	\begin{tabular}{l r}
		Tutor:      & TODO                     \\
		Instructor: & \textsc{Ullrich K\"othe} % Instructor/supervisor
	\end{tabular}
\end{center}



\newpage

\tableofcontents
\newpage

\listoffigures
\listoftables
\newpage

% If you need to include an abstract, uncomment the lines below
%\begin{abstract}
%	Abstract text
%\end{abstract}

%----------------------------------------------------------------------------------------
%	INTRODUCTION
%----------------------------------------------------------------------------------------

\section{Introduction}

%----------------------------------------------------------------------------------------
%	METHODS
%----------------------------------------------------------------------------------------

\section{Methods}

\subsection{Q-Learning}
\subsection{Reward Shaping}

%----------------------------------------------------------------------------------------
%	TRAINING
%----------------------------------------------------------------------------------------

\section{Training}
\subsection{Q-Learning}

\subsubsection*{Q Table}
A major challenge to implement Q-Learning was the creation of a feature space.
We initialize the Q table as a dictionary, whose keys are the features and the values are the values of the Q function for the six
actions.
Choosing a dictionary as the preferred data structure for the Q table provided us
two major benefits.
First, we we were able to increase its size and thus its feature space flexibly.
This proved to be quite helpful to derive training strategies because we
could focus on the q values of almost the same states.
Additionally, we didn't feel the need to create a Q table covering every possible
feature space as for unseen states, we picked actions from a similar seen state.

\subsubsection*{Feature space}
Our features consist of ten values. The majority of the values contain the Manhattan distance between
the agent and the nearest object of interest. For instance, Figure \ref{fig:example} illustrates a possible feature.
The feature shows that the agent is only one cell away from a coin in the x-axis as the first value equals to one.
The y value equals to 0, which means that the agent is in the same row as the coin.


\begin{center}
	\begin{figure}[H]
		\begin{tabular}{ccccccccccc}
			          &                                                               &                           & \multicolumn{1}{c}{\includegraphics[width=1cm]{Figures/robot_pink.png}}          &                                                             \\
			          & \multicolumn{1}{c}{\includegraphics{Figures/coin.png}}        &                           & \multicolumn{1}{c}{\includegraphics{Figures/crate.png}}                          & \multicolumn{1}{c}{\includegraphics{Figures/bomb_blue.png}} \\
			\\
			feature = & (coin x,                                              coin y, & left,  right,  down,  up, & crate x ,                                                               crate y, & bomb x,  bomb y)
		\end{tabular}
		\caption{Illustration of the feature structure encoding the space of the agent}
	\end{figure}
\end{center}


\begin{center}
	\begin{figure}[h]
		\centering
		\begin{tabular}{ccccccccccc}
			feature = & (1, & 0, & 0, & 0, & 1, & 1, & 5, & 1, & 9, & 12)
		\end{tabular}
		\caption{Example of a feature }
		\label{fig:example}
	\end{figure}
\end{center}



\subsubsection*{Auxilliary Rewards}
In order to incentivize the agent to perform beneficial actions we used a number of auxilliary rewards.
The rewards were split into five different categories, which indicate in which case the auxilliary reward
is earned. For instance, the auxilliary reward MOVED\_IN\_CYCLE corresponds to the general category. This
means that the reward can be earned at any step in the game. While the auxilliary reward MOVED\_TO\_CRATE can
only be earned when a crate or an opponent is present in the features, thus its category is Crates, Opponents.

\begin{table}[h]
	\begin{center}
		\begin{tabular}{llc}
			Category & Auxilliary Rewards          & Value \\
			\hline \hline
			\multirow{4}*{General}
			         & e.VALID\_ACTION             & 0     \\
			         & NOT\_VALID\_ACTION          & -2000 \\
			         & MOVED\_IN\_CYCLE            & -2000 \\
			         & NOT\_MOVED\_IN\_CYCLE       & 0     \\
			\hline
			\multirow{4}*{Coins}
			         & e.COIN\_COLLECTED           & 1000  \\
			         & e.COIN\_FOUND               & 20    \\
			         & MOVED\_TO\_COIN             & 50    \\
			         & MOVED\_AWAY\_FROM\_COIN     & -5    \\
			\hline
			\multirow{3}*{Crates, Opponents}
			         & e.DIDNT\_DROP\_BOMB         & -60   \\
			         & MOVED\_TO\_CRATE            & 100   \\
			         & MOVED\_AWAY\_FROM\_CRATE    & -50   \\
			\hline
			\multirow{8}*{Bombs}
			         & MOVED\_TO\_BOMB             & -2000 \\
			         & MOVED\_AWAY\_FROM\_BOMB     & 200   \\
			         & MEANINGFUL\_WAIT            & 5     \\
			         & NOT\_MEANINGFUL\_WAIT       & -30   \\
			         & SAFE\_FROM\_BOMB            & 200   \\
			         & NOT\_SAFE\_FROM\_BOMB       & -200  \\
			         & MEANINGFUL\_BOMB\_DROP      & 200   \\
			         & NOT\_MEANINGFUL\_BOMB\_DROP & -400  \\
			\hline
			\multirow{3}*{Life}
			         & e.KILLED\_SELF              & -2000 \\
			         & NOT\_KILLED\_SELF           & 0     \\
			         & e.KILLED\_OPPONENT          & 1000  \\
			         & e.OPPONENT\_ELIMINATES      & 50    \\
			         & e.GOT\_KILLED               & -2000 \\
			         & e.SURVIVER\_ROUND           & 0     \\
			\hline
		\end{tabular}
		\caption{Auxilliary rewards and their values}
		\label{tab:rewards}
	\end{center}
\end{table}



\subsubsection{Discussion}
The agent showed some promising results in specific circumstances. The following points discuss
some problems and solutions to improve the agent.

\paragraph*{Zero values}
A major bottleneck in the development of the agent was the lengthy training. The majority of
the last week before the submission was utilized for training. However, this proved to not be enough as 50\%
of the actions in the Q table contained a zero value, meaning that the action was never tried out.

\paragraph*{Better choice among zero values}
Based on the observed training per state, which equals to XX min /per XX rounds or games it wouldn't
be feasible to train the agent until it converges to a good policy. During training fine-grained
tasks such as the collection of coins, we observed that the agent showed acceptable performance
after the Q table passed the threshold of XX non-zero values. This indicates that after XX state-action pairs
were selected at least one, the Q table converged to a good policy.
When the threshold is reached, there are XX zero values in the Q table or in other words, XX state-action pairs not
tried out yet. In certain states, the tried out actions contained negativ Q values. This drove the agent to
choose one of the not tried out actions, as their value of zero is higher than any negative value.
In this case, the agent would choose the first zero valued action. An interesting adjustment, would
be to choose a random action of the zero valued. This would provide two major benefits.

1. As during axploration, we choose the first zero valued action to try, we can see that the
majority of the first actions have been tried out, while the majority of the last two actions contain the most zero values.
This drove the agent to choose the last two actions, 'WAIT' and 'BOMB', the majority of the time
when tried out actions of the current state contained a negative Q value.

\paragraph*{Decision time at unseen states}
\paragraph*{Similar states}
A major problem that we oversaw was that the agent needed more time than specified to choose an action.
This was the case, when the agent was in a state, that wasn't present in the Q table. In out implementation, we
choose an action from the most similar state present in the Q table. We define state similarity as a weighted euclidean
distance. Our weight vector $w$ is $ w= (3, 3, 2, 2, 2, 2, 1, 1, 4, 4)$. The weights were
chosen based on the winning criteria of the game. As the winner is the agent with the highest score,
we chose the highest weight of 4 for the nearest bomb distance following with a weight of 3 for the nearest coin distance.
Next, we felt that a similar state should contain the same free tiles as the unseen state, to avoid invalid actions to be chosen.
Finally, we believe that the distance to a crate or opponent is not crutial for an unseen state as they don't influence the agent't score directly.
Thus we chose a weight of 1.

\paragraph*{Computation of similar states}
To compute a similar state we used an efficient function provided by the scipy library. We
tested the implementation of fine-grained tasks and always determined the best action
whithin the given time frame. However, the final submitted agent exceeded think time in multiple time steps of every round.
The reasons is that we iterated over all states of the Q table to compute the euclidean distance between
the current state. As the Q table increased dramatically in size once we merged the three tables, the iteration took longer than the provided
think time for the agent. A better approach would be to use another more efficient function provided by scipy
that handles the iteration internally.




%----------------------------------------------------------------------------------------
%	EXPERIMENTS
%----------------------------------------------------------------------------------------

\section{Experiments and Results}

\subsection{Q-Learning}
Although our agents spends the majority of the time waiting as it exceeds the
think time, the bug is easily solved by removing iterations over the states.
As the bug was easily solved and isn't directly related to Q-Learning, the
following charts show the agent's performance after solving the bug. By removing the
minor bug, we illustrate the performance of our Q-Learning implementation.

\begin{figure}[H]
	\centering
	\includegraphics[scale=0.6]{Figures/metrics.png}
	\label{img:metrics}
	\caption{Game Metrics comparison in classic and coin-heaven scenario.}
\end{figure}

Figure \ref{img:metrics} ????????????????????? shows the game metrics in the classic and the coin-heaven
scenario.

less steps is coin heaven

higher score by 13x

More time in coin-heaven by over 2.5x
due to ??

\begin{figure}[H]
	\centering
	\includegraphics[scale=0.6]{Figures/game.png}
	\label{img:game}
	\caption{Performance comparison in classic and coin-heaven scenario.}
\end{figure}

Figure \ref{img:game} shows the effect of the moves in the classic and the coin-heaven
scenario. Over all four labels the classic scenario contains consistently higher values.
This indicates the worst performance in comparison to the coin-heaven scenario, which
aligns with our focus on training in the coin-heaven scenario.
This observation becomes even more obvious when considering bad choices such as invalid actions and suicides
which are higher.

In the classic scenario, as the agent is less cautious with its moves, its performs on average 24.5 moves per round.
The ratio is 5x.

Invalid actions ratio over 24x.

Bomb ratio over 6x.

suicides ratio over 2.5x

Overall, the agent shows better performance in the coin-heaven scenarion than in
the classic scenario. Our focus on the coin-heaven scenario created an agent..
The performance is consistently better over all criteria except the time.


%----------------------------------------------------------------------------------------
%	CONCLUSION
%----------------------------------------------------------------------------------------

\section{Conclusions}




%----------------------------------------------------------------------------------------
%	BIBLIOGRAPHY
%----------------------------------------------------------------------------------------

\printbibliography % Output the bibliography

%----------------------------------------------------------------------------------------

\end{document}